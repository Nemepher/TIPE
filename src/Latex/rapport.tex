\documentclass{article}
\usepackage[T1]{fontenc}
\usepackage[french]{babel}
\usepackage{pgfplots}
\usepackage[toc]{appendix}
\usepackage[nottoc]{tocbibind}
\pagestyle{headings}

\pgfplotsset{width=5cm,compat=1.16}

\title{Détection et identification d'arbre à partir d'imagerie satellite/aerienne}
\author{Augustin Albert}

\begin{document}

\maketitle
\tableofcontents

\section*{Introduction}
	\subsection*{Position du problème}
		-pourquoi vouloir faire ça, utilité/contexte	
		-1 pb extraction de données : differentes méthodes qui requièrent plus ou moins de materiel/images de qualité
		-2 pb traitement des données

	\subsection*{\'{E}tat actuel de la recherche}
		-voir papiers 

	\subsection*{Objectifs du TIPE}		
		-limitation à des images aériennes: pourquoi(moins couteux, accessible sur internet, differents modes d'aquisitions,  enjeux/difficultés
		-d'une part à concevoir ... pour detecter les a
		-d'une autre part à l'utiliser pour contruire une base de donné permetant identification ulterierure sur la base du machin learning
		-application au site du parc regional...
		
\section{Détection des houppiers}
	Intro: Méthode naive ( les preentations) de detection des zones plus lumineuses. La luminosité des arbres peut beaucoup varier sur une meme image (à moins d'avoir des images de haute qualité "prise en une seule fois" (ex papier). Une solution = Détection de blobs. 2 problemes : -différence de luminosité et différence d'échelle.

	\subsection{Laplacien du gaussien  et approche multi échelle}
		L'approche multi-échelle 
		-definition : permet d'obtenir une réponse particulière lors d'un echellon du signal. Le filtre LoG est défini comme le laplacien du filtre gaussien. (filtre = opérateur ? -> Fiche voca)
		\[G_{\sigma}:=\frac{1}{2\pi\sigma^{2}}\exp(-\frac{x^{2}+y^{2}}{2\sigma^{2}})\]
		\[{LoG}_{\sigma}:=-\frac{1}{\pi\sigma^{4}}(1-\frac{x^{2}+y^{2}}{2\sigma^{2}})\exp(-\frac{x^{2}+y^{2}}{2\sigma^{2}})\] 
		Demo : à faire 

		-schema: sa réponse à un contour
		convolution du filtre avec un signal crénau
		
		-schemas rapprochemet du blob pour differentes taille du blob 
		
		-Le filtre permet la detection de contour en recherchant les points d'annulation. On cherche ici à l'utiliser pour la détection de blob. (minimum) lorsque deux contours sont suffisament proches, la réponse au centre du blob est minimal à condition que la taille caractéristique du blob corresponde au paramètre sigma. relation demo r = racine 2 sigam carré 

		-mais la réponse de l'operateur s'attenue lorsque $\sigma$ augmente. Il est donc nécessaire de normaliser l'operateur. (Quel coef ? sigma carré) 
		
		-schemas normalisé ou non 

		-Appliquer log a differente echelle permet donc de ramener la detection de blob à la recherche d'un minimum par rapport à l'espace (centre du blob) et l'echelle (taille carctéristique) ) 

		\begin{tikzpicture}
			\begin{axis}[]	
			\addplot[ domain = -10:10,samples=100 ]{x^2 -2*x};
			\addplot[ domain = -10:10,samples=100 ]{x^3 -2*x};
			\end{axis}
		\end{tikzpicture}

	\subsection{mise en place}
	Implementaion en python
		-filtre gaussien et convolution :	
			-approximation Dog (avec demo) qui sera utilisé (partique) 
			-convolution et séparabilité du filtre de gauss (optimisation de l'algorithme de convolution) En pratique, convolution de scypy qui sera utilisé (bien plus rapide)
	
		-pyramide d'image puis difference pour réaliser une matrice numpy 3 dimensionelle dans laquelle on recherchera des minimums local ou global selon l'axe
		
		-selection des minimums :
			-utilisation de numpy
			-dans quel ordre et pourquoi ?
			-verification pour eviter la superposition (pourqoi et comment : certain ordre et on verifie que ce n'est pas dans les précedents)
	
		-selection des meilleurs paramètres (très heuristique comme méthode, nécessite des essais => la méthodes n'est pas completement automatique. (dépend de la taille caractéristique des arbres, de l'echelle choisie) donner les paramètre pour l'echelle et tout.

		facteur limitant i

	\subsection{\'{E}valuation des résultats}
		-évalutation de la complexité 
		-propres résultats
		faire tableau 3 colonnes pour les trois images différentes
		% repéré 
		-comparaison avec les résultats des papiers 

\section{Identification des espèces}
	
	\subsection{Propre algorithme ou tensorflow}
	
	\subsection{Méthodologie de construction d'une base de donné fiable}
		- trop long de faire à la main + 
		- géoportail (verif autorisation... cé) et extration sur des zones ou la couverture d'espèce est uniforme : res --- images triées en 2
	
	\subsection{entrainement et quelle type de modèle }
		-citer papier , à la main 

\section{Prolongements envisagables}
	-on obtient qu'un cercle autour des arbres
	-une methode watershed segmentation avec marqueurs que l'on à trouvé pourrait etre envisagble pour delinéer parfaitement les arbres (voir papier) 

\nocite{NatesanResNet} %à supprimer après!! 
\bibliographystyle{alpha}
\bibliography{references}

\begin{appendix}
	\section{Résultats}	
		-résultats intermédiare (pyramide de gauss)
		-les deux 
		-douglas seul 
		-feuillus seul
		-echantillon banque fourni pour feuillus
		-echantillon banque fourni pour douglas 
		-echantillon aléatoire parmis des images non deja vus

	\section{Démos}
		
	\section{algorithmes}
\end{appendix}

\end{document}


